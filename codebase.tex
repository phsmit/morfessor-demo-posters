 \documentclass[final]{beamer} % beamer 3.10: do NOT use option hyperref={pdfpagelabels=false} !
  %\documentclass[final,hyperref={pdfpagelabels=false}]{beamer} % beamer 3.07: get rid of beamer warnings
  \mode<presentation> {  %% check http://www-i6.informatik.rwth-aachen.de/~dreuw/latexbeamerposter.php for examples
    \usetheme{Aalto}    %% you should define your own theme e.g. for big headlines using your own logos 
  }
  \usepackage[english]{babel}
  \usepackage[latin1]{inputenc}

\DeclareRobustCommand{\AaltoLogoRandomize}[1][rby]{%
\setcounter{aaltologo_color}{2}\gdef\AaltoLogoColor{aaltoBlue}%
\setcounter{aaltologo_mark}{1}\gdef\AaltoLogoMark{?}%
}



  \usepackage{amsmath,amsthm, amssymb, latexsym}
  %\usepackage{times}\usefonttheme{professionalfonts}  % times is obsolete
  \usefonttheme[onlymath]{serif}
  \boldmath
  \usepackage[orientation=landscape,size=a2,scale=1.0,grid,debug]{beamerposter}                       % e.g. for DIN-A0 poster
  %\usepackage[orientation=portrait,size=a1,scale=1.4,grid,debug]{beamerposter}                  % e.g. for DIN-A1 poster, with optional grid and debug output
  %\usepackage[size=custom,width=200,height=120,scale=2,debug]{beamerposter}                     % e.g. for custom size poster
  %\usepackage[orientation=portrait,size=a0,scale=1.0,printer=rwth-glossy-uv.df]{beamerposter}   % e.g. for DIN-A0 poster with rwth-glossy-uv printer check
  % ...
  %
  \title{Morfessor 2.0: Toolkit for Statistical Morphological Segmentation -- Codebase}
  \author{Peter Smit, Sami Virpioja, Stig-Arne Gr\"onroos and Mikko Kurimo}
  \institute[Aalto University]{Aalto University}
  \date{April 28th, 2014}

\newlength{\columnheight}
\setlength{\columnheight}{105cm}


\usetikzlibrary{positioning}
\usetikzlibrary{fit}
\usetikzlibrary{calc}
\usetikzlibrary{shapes.geometric}

\definecolor{myblue}{HTML}{86B0FF}
\definecolor{mybluedark}{HTML}{3660CF}
\definecolor{myorange}{HTML}{FFCB94}
\definecolor{mygreen}{HTML}{A2C594}
\definecolor{mygrey}{HTML}{737379}

\definecolor{webgreen}{rgb}{0,.5,0}
\definecolor{webbrown}{rgb}{.6,0,0}
\definecolor{Maroon}{cmyk}{0, 0.87, 0.68, 0.32}
\definecolor{RoyalBlue}{cmyk}{1, 0.50, 0, 0}

\usepackage{textcomp}
\DeclareMathOperator*{\argmax}{arg\,max}
\DeclareMathOperator*{\argmin}{arg\,min}
\DeclareMathOperator*{\ind}{I}
\DeclareMathOperator*{\sign}{sign}

\newcommand{\vect}[1]{\mathbf{#1}}
\newcommand{\mat}[1]{\mathbf{#1}}
\newcommand{\mbf}[1]{\boldsymbol{#1}}
\newcommand{\seq}[1]{\boldsymbol{#1}}
\newcommand{\dom}[1]{\mathcal{#1}}


\newcommand{\len}[1]{\lvert#1\rvert}
\newcommand{\abs}[1]{\lvert#1\rvert}
\newcommand{\norm}[1]{\lVert#1\rVert}

\newcommand{\cost}{L}
\newcommand{\model}{\mathcal{M}}
\newcommand{\params}{\boldsymbol{\theta}}
\newcommand{\hparams}{\boldsymbol{\hat{\theta}}}
\newcommand{\tparams}{\boldsymbol{\tilde{\theta}}}
\newcommand{\paramspace}{\Theta}
\newcommand{\data}{\seq{D}}
\newcommand{\hatdata}{\seq{\hat{D}}}
\newcommand{\traindata}{\seq{D_{\txt{train}}}}
\newcommand{\testdata}{\seq{D_{\txt{test}}}}
\newcommand{\grammar}{\mathcal{G}}
\newcommand{\lexicon}{\mathcal{L}}
\newcommand{\corpus}{\textrm{corpus}}

%\newcommand{\word}{\phi^{-1}}
\newcommand{\detoken}{\phi^{-1}}
\newcommand{\token}{\phi}
\newcommand{\tokenset}{\Phi}
\newcommand{\bound}{\#}

\newcommand{\vb}{\,|\,}

\newcommand{\X}{\mat{X}}
\newcommand{\Y}{\mat{Y}}
\newcommand{\Z}{\mat{Z}}

  \begin{document}
  \begin{frame}{Morfessor 2.0: Toolkit for Statistical Morphological Segmentation -- Codebase} 
\begin{columns}

\begin{column}{.3\textwidth}

      \begin{beamercolorbox}[center,wd=\textwidth]{postercolumn}
 \begin{block}{Previous: Morfessor 1.0}
                \begin{itemize}
                \item Written in perl
		\item Limited utf-8 support
		\item Older codebase, unsuitable for extensions

                \end{itemize}          
            \end{block}
            
	\end{beamercolorbox}

      \begin{beamercolorbox}[center,wd=\textwidth]{postercolumn}
 \begin{block}{Morfessor 2.0}
                \begin{itemize}
                \item Complete rewrite
		\item Inclusion of many previously published features
		\item Extensible for new features and algorithms
                \end{itemize}          
            \end{block}
            
	\end{beamercolorbox}


      \begin{beamercolorbox}[center,wd=\textwidth]{postercolumn}
 \begin{block}{Usage}
              \begin{itemize}
              \item Library interface
                \begin{itemize}
                \item Directly use Morfessor from python scripts
                \end{itemize}
              \item Command line interface
                \begin{itemize}
                \item Run training evaluation and segmentation from the command line
                \item Almost complete coverage of Morfessor functionality
                \end{itemize}
              \end{itemize}              
            \end{block}
            
	\end{beamercolorbox}
\vfill

      \begin{beamercolorbox}[center,wd=\textwidth]{postercolumn}
 \begin{block}{Features}
              \begin{itemize}
              \item On-line training
	      \item Training speed-up with random skips
	      \item Frequency threshold and dampening for words in training data
		\item Possibility to weight training data likelihoods
		\item Optimization of data likelihood weight based on development set
		\item \textcolor{mygrey}{Optimization of data likelihood weight based on average morph type length}
		\item \textcolor{mygrey}{Optimization of data likelihood weight based on average morphs / word}
              \end{itemize}              

		{\small \textcolor{mygrey}{Items in grey will be released in Morfessor 2.1}}
            \end{block}
            
	\end{beamercolorbox}

\vfill




\end{column}

\begin{column}{.3\textwidth}

  \begin{beamercolorbox}[center,wd=\textwidth]{postercolumn}
 \begin{block}{Implementation}
              \begin{itemize}
              \item Python
		
                \begin{itemize}
                \item Runs on Python2, Python3 and Pypy interpreters
		\item Best performance: Pypy
                \end{itemize}
              \item Unit-agnostic code
                \begin{itemize}
                \item Split words into morphs, or sentences into phrases
                \end{itemize}
              \end{itemize}              
            \end{block}
            
	\end{beamercolorbox}


  \begin{beamercolorbox}[center,wd=\textwidth]{postercolumn}
 \begin{block}{Demo}
	\begin{itemize}
\item Build on top of library interface
\item Will be available online in future

\end{itemize}

            \end{block}
            
	\end{beamercolorbox}



  \begin{beamercolorbox}[center,wd=\textwidth]{postercolumn}
 \begin{block}{Distribution}
              \begin{itemize}
              \item Source code availabe at GitHub
		\item Packages available from the PYthon Package Index (Pypi)
              \item BSD Open Source License
\end{itemize}
            \end{block}
            
	\end{beamercolorbox}


  \begin{beamercolorbox}[center,wd=\textwidth]{postercolumn}
 \begin{block}{Upcoming developments}
              \begin{itemize}
		\item Morfessor 2.0 based model with morphotactic constraints (categories) 
              \item Data selection techniques
		
\end{itemize}
            \end{block}
            
	\end{beamercolorbox}
	  \begin{beamercolorbox}[center,wd=\textwidth]{postercolumn}
	 \begin{block}{Acknowledgements}
\footnotesize
The authors have received funding from the
EC's 7th Framework Programme (FP7/2007--2013)
under
grant agreement n\textdegree 287678 and the Academy of Finland under
the Finnish Centre of Excellence Program 2012--2017 (grant
n\textdegree251170) and the LASTU Programme (grants n\textdegree256887 and 259934). 
The experiments were performed using computer resources within the Aalto University School of Science "Science-IT" project.    
	            \end{block}
	            
		\end{beamercolorbox}


\end{column}
\begin{column}{.35\textwidth}


  \begin{beamercolorbox}[center,wd=\textwidth]{postercolumn}
 \begin{block}{}
\tikzstyle{databox} = [fill=myblue!50,draw, thin, text centered, text width=8em,minimum height=1.5em]
\tikzstyle{toolbox} = [draw, thin, text centered, text width =8em, rounded corners=.8ex,minimum height=3em]
\tikzstyle{modebox} = [draw, text centered, thick, text width=6em]
\tikzstyle{resubox} = [text centered, text width=8em, fill=gray!30]


\begin{tikzpicture}[scale=0.5,xscale=4,yscale=3,thick,>=latex]
\node[databox] (traindata) at (0,0) {Training data};
\node[databox, dashed] (annodata) at (4,0) {Annotation data};

\node[toolbox, text width=12em, minimum height=2em] (morftrain) at (2,-2) {\texttt{morfessor-train}};
\node[modebox] (morfmodel) at (2,-4) {Morfessor model};

\node[databox] (testdata) at (4,-6) {Corpus};
\node[databox] (goldstd) at (0,-6)  {Gold standard};

\node[toolbox] (morfsegment) at (4,-8) {\texttt{morfessor- segment}};
\node[toolbox] (morfevaluate) at (0,-8) {\texttt{morfessor- evaluate}};

\node[resubox] (segmented) at (4,-11) {Segmented corpus};
\node[resubox] (bprscore) at (0,-11) {boundary precision and recall scores};

\draw[->] (traindata) to (traindata.south |- morftrain.north);
\draw[->,dashed] (annodata) -- (annodata.south |- morftrain.north);

\draw[->] (morftrain) -- (morfmodel);

\draw[-] (morfmodel) to (2,-8);
\draw[->] (2,-8) to (morfsegment);
\draw[->] (2,-8) to (morfevaluate);

\draw[->] (testdata) to (morfsegment);
\draw[->] (goldstd) to (morfevaluate);


\draw[->] (morfsegment) to (segmented);
\draw[->] (morfevaluate) to (bprscore);

\end{tikzpicture}
            \end{block}
            
	\end{beamercolorbox}



	  \begin{beamercolorbox}[center,wd=\textwidth]{postercolumn}
	 \begin{block}{Links}
	              \begin{itemize}
			\item Homepage \\
	                \url{http://www.cis.hut.fi/projects/morpho/}

	\item GitHub\\
\url{https://github.com/aalto-speech/morfessor} 


	              \item Pypi\\ \texttt{pip install morfessor} \\
\url{https://pypi.python.org/pypi/Morfessor} 


	              \item Documentation 
	                \\ {\small \url{http://morfessor.readthedocs.org/en/latest/} }
	                
%	
%			
%	
	              \end{itemize}              
	            \end{block}
	            
		\end{beamercolorbox}








\end{column}

\end{columns}
 \end{frame}



  \end{document}
  